\documentclass[12pt, a4paper]{article}

\immediate\write18{cat src/*.tex | detex | wc -m > cc.tex}

\usepackage[utf8]{inputenc}
\usepackage[T1]{fontenc}

\usepackage[ngerman]{babel}

\usepackage[doublespacing]{setspace}
\usepackage{csquotes}

%footcite
\usepackage{biblatex}
\usepackage{hyperref}

%Linenumbers
%http://texblog.org/2012/02/08/adding-line-numbers-to-documents/
\usepackage{lineno}

\usepackage{geometry}
%\usepackage{showframe}

\usepackage{hanging}
\usepackage[flushmargin]{footmisc}
%\setlength\footnotemargin{8pt}

%Für Fussnoten \fn{}
\renewcommand{\footnotemargin}{8pt}
\newcommand{\fn}[1]{\footnote{\hangpara{2em}{1} #1}}
%http://tex.stackexchange.com/questions/15952/layout-of-multiple-lines-footnotes
%http://tex.stackexchange.com/questions/65849/confusion-onehalfspacing-vs-spacing-vs-word-vs-the-world

\begin{document}

    %Titleseite
    \newgeometry{left=2cm, right=2cm, top=2cm, bottom=2cm}
    \begin{titlepage}
	\begin{onehalfspace}
		\raggedright
		{\textsc\normalsize Universität Bern\par}
		{\textsc\normalsize Institut für Germanistik\par}
		{\textsc\normalsize (Seminar)\par}
		{\textsc\normalsize (Frühlingsemester/Herbstsemester yyyy)\par}
		{\textsc\normalsize Leitung: (Leitung)\par}

		\vfill
		\centering
		{\LARGE Titel der Arbeit\par}
		{\large Untertitel der Arbeit\par}	
		\vfill
		\raggedright
		{\textsc\normalsize (Vorname Nachname)\par}	
		{\textsc\normalsize (Adresse)\par}	
		{\textsc\normalsize (Martrikelnummer)\par}	
		{\textsc\normalsize (Email)\par}	
		\vspace{1cm}
		{\textsc\normalsize Bachelor of Arts in German Language and Literature\par}	
		{\textsc\normalsize Minor: Computer Science\par}	
		{\textsc\normalsize (Semester)\par}	
		\vspace{1cm}
		{\textsc\normalsize Anzahl Zeichen: \input{cc.tex}\par}	
	\end{onehalfspace}
\end{titlepage}

    \newgeometry{left=2cm, right=4cm, top=3cm, bottom=3cm}

    %Inhaltsverzeichnis
    \tableofcontents
    \newpage

    \section{Einleitung}
	Lorem ipsum dolor sit amet, consetetur sadipscing elitr, sed diam nonumy eirmod tempor invidunt ut labore et dolore magna aliquyam erat, sed diam voluptua. At vero eos et accusam et justo duo dolores et ea rebum. Stet clita kasd gubergren, no sea takimata sanctus est Lorem ipsum dolor sit amet. Lorem ipsum dolor sit amet, consetetur sadipscing elitr, sed diam nonumy eirmod tempor invidunt ut labore et dolore magna aliquyam erat, sed diam voluptua. At vero eos et accusam et justo duo dolores et ea rebum. Stet clita kasd gubergren, no sea takimata sanctus est Lorem ipsum dolor sit amet.

    \newpage

    %Hauptteil
    \section{Hauptteil}
    Lorem ipsum dolor sit amet, consetetur sadipscing elitr, sed diam nonumy eirmod tempor invidunt ut labore et dolore magna aliquyam erat, sed diam voluptua. At vero eos et accusam et justo duo dolores et ea rebum. Stet clita kasd gubergren, no sea takimata sanctus est Lorem ipsum dolor sit amet. Lorem ipsum dolor sit amet, consetetur sadipscing elitr, sed diam nonumy eirmod tempor invidunt ut labore et dolore magna aliquyam erat, sed diam voluptua. At vero eos et accusam et justo duo dolores et ea rebum. Stet clita kasd gubergren, no sea takimata sanctus est Lorem ipsum dolor sit amet.
    \newpage
    
    \section{Schluss}
    Lorem ipsum dolor sit amet, consetetur sadipscing elitr, sed diam nonumy eirmod tempor invidunt ut labore et dolore magna aliquyam erat, sed diam voluptua. At vero eos et accusam et justo duo dolores et ea rebum. Stet clita kasd gubergren, no sea takimata sanctus est Lorem ipsum dolor sit amet. Lorem ipsum dolor sit amet, consetetur sadipscing elitr, sed diam nonumy eirmod tempor invidunt ut labore et dolore magna aliquyam erat, sed diam voluptua. At vero eos et accusam et justo duo dolores et ea rebum. Stet clita kasd gubergren, no sea takimata sanctus est Lorem ipsum dolor sit amet.
    \newpage

    \section{Literaturverzeichnis}
    \subsection*{Textausgabe}
\begin{hangparas}{.5in}{1}
	Goethe, Johann Wolfgang: \textit{Hermann und Dorothea},
		Stuttgart: Reclam,
		\textsuperscript{4}2013 (Reclams Universal-Bibliothek, Bd. 55).
\end{hangparas}

\subsection*{Forschungsliteratur}

\begin{hangparas}{.5in}{1}
Gretz, Daniela: \enquote*{Quelle / Brunnen},
	in: Günter Butzer und Joachim Jacob (Hg.):  Metzler Lexikon literarische Symbole,
	Stuttgart und Weimar: Metzler, \textsuperscript{2}2012, S.322.
\end{hangparas}

\end{document}
